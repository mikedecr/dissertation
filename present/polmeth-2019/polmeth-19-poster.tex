% --------------------------------------
% % a0poster Portrait Poster
% LaTeX Template
% Version 1.0 (22/06/13)
%
% The a0poster class was created by:
% Gerlinde Kettl and Matthias Weiser (tex@kettl.de)
%
% This template has been downloaded from:
% http://www.LaTeXTemplates.com
%
% License:
% CC BY-NC-SA 3.0 (http://creativecommons.org/licenses/by-nc-sa/3.0/)
% --------------------------------------





\documentclass[a0]{a0poster}

% This is so we can have multiple columns of text side-by-side
\usepackage{multicol}
% This is the amount of white space between the columns in the poster
\columnsep=100pt
% This is the thickness of the black line between the columns in the poster
\columnseprule=2pt


\usepackage{booktabs} % Top and bottom rules for table
\usepackage[font=small,labelfont=bf]{caption} % Required for specifying captions to tables and figures
\usepackage{amsmath} % For math fonts, symbols and environments
\usepackage{wrapfig} % Allows wrapping text around tables and figures
\usepackage{subcaption}



% --------------------------------------
%  My packages
% --------------------------------------



\usepackage{graphicx}
\usepackage{float}
\usepackage{placeins}
\usepackage{fontawesome}

\usepackage{sectsty} % section styles
  % \allsectionsfont{\Huge \sffamily \raggedright}
  \sectionfont{\huge \sffamily \uppercase }
  \subsectionfont{\LARGE \sffamily}
  \paragraphfont{\Large \sffamily \bfseries}

\usepackage[lf, mathtabular, minionint]{MinionPro}
\usepackage[scaled = 0.9]{FiraSans}
\usepackage[varqu, scaled = 0.95]{zi4}             % mono w/ straight quotes



% \usepackage{natbib}
%   \bibpunct[: ]{(}{)}{;}{a}{}{,}

\usepackage[authordate, backend = biber]{biblatex-chicago}
\addbibresource{bib/poster-bib.bib}

\usepackage[usenames,dvipsnames]{xcolor} % extra colors
\usepackage{hyperref}
  \hypersetup{
    colorlinks = true, 
    citecolor = NavyBlue, 
    linkcolor = red, 
    urlcolor = NavyBlue
  }

\usepackage{parskip}

\usepackage{enumitem}

\usepackage[margin = 1in]{geometry}


\begin{document}


% --------------------------------------
%  HEADING - two sections
% --------------------------------------

% --- heading 1.1: metadata (title, author, etc) -----------------

\begin{minipage}[b]{0.6\linewidth}

{\VERYHuge \textsf{\textbf{Do Primaries Work?}}} \\[24pt]
{\veryHuge \emph{Local Policy Conservatism and Primary Candidate Positioning}} \\[24pt]
\Huge \textsf{\textbf{Michael G.\ DeCrescenzo}}
% $\,$  University of Wisconsin--Madison
\huge 
$ \big[\text{\faSafari $\,$ \href{https://mikedecr.github.io}{\sffamily mikedecr.github.io}}\big]
\big[\text{\faGithub $\,$ \href{https://github.com/mikedecr}{\sffamily mikedecr}}\big]
\big[\text{\faTwitter $\,$ \href{https://twitter.com/mikedecr}{\sffamily mikedecr}}\big]$

\end{minipage}
%
%
%%%%%%%%%%%%%%%%%%%%%%%%%%%%%%%%%%%%%%%%
% section 2
%%%%%%%%%%%%%%%%%%%%%%%%%%%%%%%%%%%%%%%%
% %
\begin{minipage}[b]{0.33\linewidth}
\raggedleft \includegraphics[width=0.7\textwidth]{graphics/color-flush-UWlogo-print-eps-converted-to.pdf}
\end{minipage}



\vspace{2em} % A bit of extra whitespace between the header and poster content
%
% %----------------------------------------------------------------------------------------
%
%
%
%
%
%
%
% This is how many columns your poster will be broken into, a portrait poster is generally split into 2 columns
\begin{multicols*}{3}

\Large
\raggedright
% \section*{Overview}

\raggedcolumns

\section*{Mass Representation in Party Nominations}

Political scientists believe that party nominations present candidates with a "strategic positioning dilemma'' \parencite{brady-han-pope:2007:out-of-step}: in order to win the nomination and the general election, candidates position themselves as a spatial compromise between their partisan base and the median voter. \textsf{\textbf{How do we know that voters' policy preferences matter in nominations?}}

\paragraph{Theoretical issues:}
\begin{itemize}
  \item Nominating elections have high information costs; do voters meet them?
  \item Interest groups/informal party networks may supersede voter's policy preferences.
\end{itemize}

\paragraph{Methodological issues:}
\begin{itemize}
  \item Lack direct measures of voter preferences in nominating constituencies.
  \item Existing studies of primary representation operationalize voter preferences using insufficient proxies (can't infer ideal points from vote shares, \cite{kernell2009giving}) or don't operationalize constituency preferences at all.
\end{itemize}

\vspace{1em}



% As general election competition in the U.S. becomes increasingly driven by partisanship, party nomination contests are perhaps more consequential than ever. Yet research on U.S. elections knows little about whether primaries serve their stated purpose: to represent the preferences of local partisan voters.

% theory
% existing research
% problems w/ readiily available measures

\begin{center}
  \includegraphics[width = \linewidth]{graphics/two-parties.pdf}
\end{center}

% \vfill 

\section*{Questions}

\begin{enumerate}
  \item How to measure policy preferences of \textbf{\emph{distinct partisan groups}} in the \emph{\textbf{same district}}?
  \item Do candidates position themselves to fit \emph{\textbf{party-public preferences}}?
\end{enumerate}



\columnbreak



\section*{IRT Model for District Partisan Preferences}

\paragraph{Group-level model:}
Estimate ideal points for \textbf{\emph{partisan groups in each district}}. Assume individual ideal points $\theta_{i}$ are Normal within district-party groups $g$.
\begin{align}
  \theta_{i} &\sim \mathrm{Normal}\left( \bar{\theta}_{g[i]} , \sigma_{g[i]} \right)
\end{align}

Estimate $\bar{\theta}_{g}$ with group-level item response probit model \parencite{caughey-warshaw:2015:DGIRT}.
\begin{align}
  \mathrm{Pr}\left( y_{\mathit{ij}} \right) &= 
    \Phi\left( 
      \frac{
        \bar{\theta}_{g[i]} - \kappa_{j}
      }{
        \sqrt{ \sigma_{g[i]}^{2} + \sigma^{2}_{j} }
      } 
    \right)
\end{align}

Group means pooled toward hierarchical regression. Data from districts ($d$), states ($s$), and regions ($r$) with \emph{\textbf{party-specific parameters}} ($p$):
\begin{align}
  \bar{\theta}_{g} &\sim \mathrm{Normal}\left(\zeta_{p} + \mathbf{x}^{'}_{d}\beta_{p} + \alpha_{sp} + \psi_{rp}, \sigma^{\texttt{district}}_{p} \right) 
  % \\[12pt]
  % \alpha_{sp} &\sim \mathrm{Normal}\left( Z_{s}\gamma_{p}, \sigma_{p}^{\texttt{state}} \right) \\[12pt]
  % \psi_{rp} &\sim \mathrm{Normal}\left(0, \sigma_{p}^{\texttt{region}}\right)
\end{align}


% \section*{Data}

\textsf{\textbf{Opinion data:}} CCES (2012, 2014, 2016), ANES (2012, 2016) \\
\textsf{\textbf{Covariate data:}} \textcite{foster-molina:2016:data}, Correlates of State Policy

% \columnbreak

\subsection*{Model Estimates}

\begin{center}
  \includegraphics[width = 0.49\linewidth]{graphics/pts.pdf}
  \includegraphics[width = 0.49\linewidth]{graphics/scatter.pdf}
\end{center}

% \begin{wrapfigure}{l}{0.6\linewidth}
% \begin{center}
%   \includegraphics[width = \linewidth]{graphics/scatter.pdf}
% \end{center}
% \end{wrapfigure}

Non-relationship between partisans in same district justifies modeling decisions:
\begin{itemize}
  \item Estimating parties separately 
  \item Allowing hierarchical parameters to vary by party
\end{itemize}

\columnbreak

\section*{Relationship to Candidate Positions}

\emph{\textbf{Initial exploration}} suggests district-party preferences are positively but weakly related to candidate positioning (CF scores)\ldots

\begin{center}
  \includegraphics[width = 0.9\linewidth]{graphics/pos.pdf}
\end{center}

\ldots but district-party preferences are \emph{\textbf{stronger predictors}} than district presidential vote.

\vspace{1em}

\begin{center}
  \includegraphics[width = \linewidth]{graphics/coefs.pdf}
\end{center}

% \columnbreak


%
% %%%%%%%%%%%%%%%%%%%%%%%%%%%%%%%%%%%%%%%%
% % References
% %%%%%%%%%%%%%%%%%%%%%%%%%%%%%%%%%%%%%%%%
\vfill
\footnotesize
\printbibliography
% %delete me?%\newpage
% \bibliographystyle{/Users/michaeldecrescenzo/Dropbox/apsr2006.bst}
% \bibliography{/Users/michaeldecrescenzo/Dropbox/bib.bib}

\end{multicols*}



\end{document}
