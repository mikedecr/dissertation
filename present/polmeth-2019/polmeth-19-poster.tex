% --------------------------------------
% % a0poster Portrait Poster
% LaTeX Template
% Version 1.0 (22/06/13)
%
% The a0poster class was created by:
% Gerlinde Kettl and Matthias Weiser (tex@kettl.de)
%
% This template has been downloaded from:
% http://www.LaTeXTemplates.com
%
% License:
% CC BY-NC-SA 3.0 (http://creativecommons.org/licenses/by-nc-sa/3.0/)
% --------------------------------------





\documentclass[a0]{a0poster}

% This is so we can have multiple columns of text side-by-side
\usepackage{multicol}
% This is the amount of white space between the columns in the poster
\columnsep=100pt
% This is the thickness of the black line between the columns in the poster
\columnseprule=2pt


\usepackage{booktabs} % Top and bottom rules for table
\usepackage[font=small,labelfont=bf]{caption} % Required for specifying captions to tables and figures
\usepackage{amsmath} % For math fonts, symbols and environments
\usepackage{wrapfig} % Allows wrapping text around tables and figures



% --------------------------------------
%  My packages
% --------------------------------------



\usepackage{graphicx}
\usepackage{float}
\usepackage{placeins}

\usepackage{sectsty} % section styles
  % \allsectionsfont{\Huge \sffamily \raggedright}
  \sectionfont{\huge \sffamily \uppercase }
  \paragraphfont{\LARGE \sffamily \bfseries}

\usepackage[lf, mathtabular, minionint]{MinionPro}
\usepackage[scaled = 0.9]{FiraSans}
\usepackage[varqu, scaled = 0.95]{zi4}             % mono w/ straight quotes



% \usepackage{natbib}
%   \bibpunct[: ]{(}{)}{;}{a}{}{,}

\usepackage[authordate, backend = biber]{biblatex-chicago}
\addbibresource{/Users/michaeldecrescenzo/Dropbox/bib.bib}

\usepackage[usenames,dvipsnames]{xcolor} % extra colors
\usepackage{hyperref}
  \hypersetup{
    colorlinks = true, 
    citecolor = NavyBlue, 
    linkcolor = red, 
    urlcolor = NavyBlue
  }

\usepackage{parskip}

\usepackage{enumitem}

\usepackage[margin = 1in]{geometry}


\begin{document}


% --------------------------------------
%  HEADING - two sections
% --------------------------------------

% --- heading 1.1: metadata (title, author, etc) -----------------

\begin{minipage}[b]{0.6\linewidth}

{\VERYHuge \textsf{\textbf{Do Primaries Work?}}} \\[24pt]
{\veryHuge \emph{Local Policy Liberalism and Primary Candidate Positioning}} \\[24pt]
\Huge \textsf{\textbf{Michael G.\ DeCrescenzo}}
$\,$ \Huge University of Wisconsin--Madison

\end{minipage}
%
%
%%%%%%%%%%%%%%%%%%%%%%%%%%%%%%%%%%%%%%%%
% section 2
%%%%%%%%%%%%%%%%%%%%%%%%%%%%%%%%%%%%%%%%
% %
\begin{minipage}[b]{0.33\linewidth}
\centering \includegraphics[width=0.6\textwidth]{graphics/UWlogo.png}
\vfill
\end{minipage}



\vspace{2cm} % A bit of extra whitespace between the header and poster content
%
% %----------------------------------------------------------------------------------------
%
%
%
%
%
%
%
% This is how many columns your poster will be broken into, a portrait poster is generally split into 2 columns
\begin{multicols*}{3}

\LARGE
\raggedright
% \section*{Overview}

\raggedcolumns

\section*{Background}

\columnbreak



\section*{IRT Model for District Party-Public Liberalism}

\paragraph{Group-level model:}
Estimate ideal points for \textbf{\emph{partisan groups in each district}}. Assume individual ideal points $\theta_{i}$ are Normal within district-party groups $g$.
\begin{align}
  \theta_{i} &\sim \mathrm{Normal}\left( \bar{\theta}_{g[i]} , \sigma_{g[i]} \right)
\end{align}

Estimate $\bar{\theta}_{g}$ with group-level item response model \parencite{caughey2015dynamic}.
\begin{align}
  \mathrm{Pr}\left( y_{\mathit{ij}} \right) &= 
    \Phi\left( 
      \frac{
        \bar{\theta}_{g[i]} - \kappa_{j}
      }{
        \sqrt{ \sigma_{g[i]}^{2} + \sigma^{2}_{j} }
      } 
    \right)
\end{align}

Group means smoothed with hierarchical model. Data from districts ($d$), states ($s$), and regions ($r$) with \emph{\textbf{party-specific parameters}} ($p$):
\begin{align}
  \bar{\theta}_{g} &\sim \mathrm{Normal}\left(\zeta_{p} + X_{d}\beta_{p} + \alpha_{sp} + \psi_{rp}, \sigma^{\texttt{district}}_{p} \right) \\[12pt]
  \alpha_{sp} &\sim \mathrm{Normal}\left( Z_{s}\gamma_{p}, \sigma_{p}^{\texttt{state}} \right) \\[12pt]
  \psi_{rp} &\sim \mathrm{Normal}\left(0, \sigma_{p}^{\texttt{region}}\right)
\end{align}


\section*{Data}

Survey data from CCES (2012, 2014, 2016), ANES (2012, 2014, 2016)

\columnbreak




\section*{Results}

\begin{center}
  \includegraphics[width = 0.7\linewidth]{graphics/histograms.pdf}
\end{center}

% \columnbreak


%
% %%%%%%%%%%%%%%%%%%%%%%%%%%%%%%%%%%%%%%%%
% % References
% %%%%%%%%%%%%%%%%%%%%%%%%%%%%%%%%%%%%%%%%
\vfill
\footnotesize
\printbibliography
% %delete me?%\newpage
% \bibliographystyle{/Users/michaeldecrescenzo/Dropbox/apsr2006.bst}
% \bibliography{/Users/michaeldecrescenzo/Dropbox/bib.bib}

\end{multicols*}



\end{document}
